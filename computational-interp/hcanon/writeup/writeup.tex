\documentclass[10pt]{article}

\usepackage{fullpage}
\usepackage{times}
\usepackage{multicol}
\usepackage{mathptmx}
\usepackage{color}
\usepackage[cmex10]{amsmath}
\usepackage{amsthm}
\usepackage{amssymb}
\usepackage{stmaryrd}
\usepackage[square,numbers,sort]{natbib}
\usepackage{arydshln}
\usepackage{graphics}
\usepackage{natbib}
\usepackage{url}
\usepackage{lagdatotex}
\usepackage{relsize}

\usepackage{tikz}
\usetikzlibrary{decorations.pathmorphing}

\usepackage{ucs}
\usepackage[utf8x]{inputenc}
\usepackage[T1]{fontenc}

\newcommand{\noagda}[0]{}
                                                                                                                                                 
\usepackage{graphicx}                                                                                                                                          
\newcommand\Shamrock{\includegraphics[width=0.6em]{sham}}                                                                                                      
\newcommand\Shamrockv[1]{\ensuremath{\includegraphics[width=0.6em]{sham}_{#1}}}  

\DeclareUnicodeCharacter{8594}{$\shortrightarrow$}
\DeclareUnicodeCharacter{9001}{$\langle$}
\DeclareUnicodeCharacter{9002}{$\rangle$}
\DeclareUnicodeCharacter{12314}{$\llbracket$}
\DeclareUnicodeCharacter{12315}{$\rrbracket$}
\DeclareUnicodeCharacter{8872}{$\vDash$}
\DeclareUnicodeCharacter{9711}{\ensuremath{\bigcirc}}
\DeclareUnicodeCharacter{9675}{$\circ$}
\DeclareUnicodeCharacter{9671}{$\Diamond$}
\DeclareUnicodeCharacter{9657}{$\triangleright$}
\DeclareUnicodeCharacter{8640}{$\rightharpoonup$}
\DeclareUnicodeCharacter{8659}{$\Downarrow$}
\DeclareUnicodeCharacter{8984}{$\Shamrock$}

\newcommand\textPsi{$\Psi$}
\newcommand\textXi{$\Xi$}
\newcommand\textxi{$\xi$}
\newcommand\textDelta{$\Delta$}
\newcommand\textGamma{$\Gamma$}
\newcommand\textsigma{$\sigma$}
\newcommand\textSigma{$\Sigma$}
\newcommand\textPi{$\Pi$}
\newcommand\textrho{$\rho$}
\newcommand\textphi{$\varphi$}
\newcommand\texttau{$\tau$}
\newcommand\texteta{$\eta$}
\newcommand\textalpha{$\alpha$}
\newcommand\textbeta{$\beta$}
\newcommand\textepsilon{$\epsilon$}
\newcommand\textkappa{$\kappa$}
\newcommand\textOmega{$\Omega$}
\newcommand\textmho{$\mho$}
\newcommand\textgamma{$\gamma$}
\newcommand\textlambda{$\lambda$}
\newcommand\textpi{$\pi$}
\newcommand\texttheta{$\theta$}
\newcommand\textdelta{$\delta$}

\usepackage{fancyvrb}

\newcommand{\ttt}[1]{\texttt{#1}}

\newcommand{\ignore}[1]{}

\hyphenation{func-tor-ially}

%% disguise from Agda 
\newenvironment{noagdacode}
  {\begin{code}}
  {\end{code}}

\newcommand\ecmd[1]{\textbf{#1}}

\begin{document}

\title{Homotopy Canoncity by Logical Relations: HSets}


% author names and affiliations
% use a multiple column layout for up to three different
% affiliations
\author{DRL \qquad CA \qquad RWH}

\maketitle

If this all works, I'm pretty psyched about it:

\begin{enumerate}

\item It gives an operational semantics for hset-type-theory, and maybe
  for 1-type-type-theory as well (see below).  

\item We can run it in Agda, so it could be used as the basis for a
  tactic library.  Of course, there would be a lot of work quoting terms
  written in full Agda into the little embedded type theory that the
  proof works for.  

\item It gives a new justification for function extensionality, in terms
  of regular (Per-approved) intensional type theory.  Related work: There's the
  Hofmann setoid model, which doesn't model everything, and the
  Altenkirch one / OTT, which requires proof irrelvance.  Here, we'll
  show that pure intensional type theory (with induction-recursion,
  etc. but without any funny business in the identity type) can prove
  canonicity (and consistency?) for type theory with funext.  

\item I had an idea in the car yesterday about the puzzling
  ``off-by-one'' phenomenon in the Coquand/Barras work.  

  Consider hset-type-theory, where all types are hsets, with a univalent
  universe of hprops.  I know of three proofs of canonicity for this:

  Coquand/Barras interpret types as simplicial sets, but they always go
  one level higher than the type theory itself.  In this case, they
  interpret a type as (essentially) a type as a setoid, so it really has
  points and paths.

  In contrast, the Shulman proof for hset-type-theory interprets types
  as things-glued-with-sets, so a type is more like a set than a setoid.

  There's also the OTT datapoint, which justifies hset-type-theory
  essentially by interpreting types as sets---this comes out in the fact
  that the way they do things, there aren't any constraints on paths,
  other than consistency.  

  I have a guess about what's going on: to prove canonicity (like in
  Coquand/Barras), you need the extra level.  To prove canonicity
  \emph{from consistency}, you don't (like in OTT).  We can ask Mike
  whether he used consistency to test my claim.  

  The upshot for this work is that I think we should be able to prove
  canonicity for hset-type-theory, or prove canonicity-from-consistency
  for 1-type-type-theory, using pretty much the same proof.  That is, so
  far, I don't see a spot where the proof below would break if we put
  Bool in the unverse, except that we'd need to assume that there is no
  proof of true = false.  On the other hand, I think that we could
  change $Q_{bool}$ in such a way that we wouldn't need consistency---a
  path is reducible at bool the endpoints are both true or both false,
  but not if one is true and the other is false.  But we couldn't make
  the corresponding definition if bool were in the universe (for the
  current notion of candidate, which doesn't allow you to specify a
  relation for the identity type of things in the universe).

  If the details on this work out, it would placate my confusion about
  this issue.
\end{enumerate}

\input{writeup0811.lao}

\end{document}

