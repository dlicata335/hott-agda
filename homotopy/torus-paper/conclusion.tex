\section{Related and Future Work}

The various definitions of path-over, and their equivalence, were
discussed during the IAS special year (see \citep[Remark
  6.3.2]{uf13hott-book}).  The idea to use inductive families to define
shapes other than lines was explored in a simplicial setting by Coquand
(e.g. a triangle in a type).  At the time, it was considered an open
question whether it was helpful to consider shapes other than the
homogeneous globes that arise by iterating the identity type.  In this
paper, we have argued that there are benefits to working with cube and
cube-over types.  Higher cubes and cube-overs arise naturally when
higher inductive types have two-dimensional (or higher) constructors
(like the torus), when higher inductive types are nested (a pushout of
pushouts, like in the three-by-three lemma), and when eliminations on
higher inductive types are nested (like when mapping out of two
circles).  Developing lemmas in terms of these abstractions has enabled
the new formalizations described in this paper.

At present, we have developed cube types in ``offshoots'' of Agda
homotopy type theory libraries; an interesting direction for future work
would be to revise the libraries to use cubes throughout, and to revisit
existing results to see if they can be simplified.  Another direction
for future work is to define higher cube and cube-over types in a
dimension-polymorphic way, rather than implementing each dimension in
isolation, as we have done here.

Our work is carried out in the setting of dependent type theory with
axioms for univalence and higher inductives, but there have also been
extensions of dependent type theory based on cubical ideas.  These
include a type theory with a computational interpretation of
parametricity~\citep{bernardy12parametricity}, and new cubical type
theories~\citep{coquand14variations,altenkirchkaposi14cubical,lb14cubes-oxford}
inspired by the cubical sets model of homotopy type
theory~\citep{coquand+13cubical}.  These cubical type theories add new
definitional equalities to the axiomatic formulations of univalence and
higher inductives.  The examples presented here could be translated to
cubical type theories, to investigate the impact of these new
definitional equalities.  Additionally, it is an open question whether
results in these stricter cubical type theories can be translated back
to the axiomatic type theory, by inserting more propositional reasoning;
the low-dimensional libraries and examples presented here provide a bit
of empirical evidence in the positive direction.
