\documentclass[conference, compsocconf]{drl-common/IEEEtran}
\IEEEoverridecommandlockouts

\usepackage{multicol}
\usepackage{mathptmx}
\usepackage{color}
\usepackage[cmex10]{amsmath}
\usepackage{amsthm}
\usepackage{amssymb}
\usepackage{stmaryrd}
\usepackage{drl-common/proof}
\usepackage{drl-common/typesit}
\usepackage{drl-common/typescommon}
\usepackage[square,numbers,sort]{natbib}
%% \usepackage{arydshln}
\usepackage{graphics}
\usepackage{natbib}
\usepackage{url}
\usepackage{drl-common/lagdatotex}
\usepackage{relsize}

\usepackage{tikz}
\usetikzlibrary{decorations.pathmorphing}

                                                                                                                                                 
\usepackage{graphicx}                                                                                                                                          
\newcommand\Shamrock{\includegraphics[width=0.6em]{sham}}                                                                                                      
\newcommand\Shamrockv[1]{\ensuremath{\includegraphics[width=0.6em]{sham}_{#1}}}  

\newcommand{\noagda}[0]{}
\usepackage{ucs}
\usepackage[utf8x]{inputenc}
\usepackage[T1]{fontenc}
\DeclareUnicodeCharacter{8594}{$\shortrightarrow$}
\DeclareUnicodeCharacter{9001}{$\langle$}
\DeclareUnicodeCharacter{9002}{$\rangle$}
\DeclareUnicodeCharacter{12314}{$\llbracket$}
\DeclareUnicodeCharacter{12315}{$\rrbracket$}
\DeclareUnicodeCharacter{8872}{$\vDash$}
\DeclareUnicodeCharacter{9711}{\ensuremath{\bigcirc}}
\DeclareUnicodeCharacter{9675}{$\circ$}
\DeclareUnicodeCharacter{9671}{$\Diamond$}
\DeclareUnicodeCharacter{9657}{$\triangleright$}
\DeclareUnicodeCharacter{8640}{$\rightharpoonup$}
\DeclareUnicodeCharacter{8659}{$\Downarrow$}
\DeclareUnicodeCharacter{8984}{$\Shamrock$}
\DeclareUnicodeCharacter{8596}{$\leftrightarrow$}
\DeclareUnicodeCharacter{7522}{${}_i$}
\DeclareUnicodeCharacter{11388}{$\!{}_j$}
% Bullet and box, the heights are adjusted to make it look nice
\DeclareUnicodeCharacter{8226}{\raisebox{-1.6pt}{$\bullet$}}
\DeclareUnicodeCharacter{9726}{$\rule[-1pt]{3pt}{3pt}$}

\newcommand\textPsi{$\Psi$}
\newcommand\textXi{$\Xi$}
\newcommand\textxi{$\xi$}
\newcommand\textDelta{$\Delta$}
\newcommand\textGamma{$\Gamma$}
\newcommand\textsigma{$\sigma$}
\newcommand\textSigma{$\Sigma$}
\newcommand\textPi{$\Pi$}
\newcommand\textrho{$\rho$}
\newcommand\textphi{$\varphi$}
\newcommand\texttau{$\tau$}
\newcommand\texteta{$\eta$}
\newcommand\textalpha{$\alpha$}
\newcommand\textbeta{$\beta$}
\newcommand\textepsilon{$\epsilon$}
\newcommand\textkappa{$\kappa$}
\newcommand\textOmega{$\Omega$}
\newcommand\textmho{$\mho$}
\newcommand\textgamma{$\gamma$}
\newcommand\textlambda{$\lambda$}
\newcommand\textpi{$\pi$}
\newcommand\texttheta{$\theta$}
\newcommand\textdelta{$\delta$}

\usepackage{fancyvrb}
\newcommand{\ttt}[1]{\texttt{#1}}

\usepackage{drl-common/code}
\DefineVerbatimEnvironment{code}{Verbatim}{fontsize=\small,fontfamily=tt}

\newcommand{\ignore}[1]{}

%% small tightcode, with space around it
\newenvironment{stcode}
{\smallskip
\begin{small}
\begin{tightcode}}
{\end{tightcode}
\end{small}
\smallskip}



\newcommand{\sone}{{\dsd{S^1}}}
\newcommand{\base}{{\dsd{base}}}
\newcommand{\lp}{{\dsd{loop}}}

\renewcommand{\id}[3]{\ensuremath{\dsd{Id}_{#1}(#2, #3)}}

\renewcommand{\inl}[1]{\app{\dsd{inl}}{#1}}
\renewcommand{\inr}[1]{\app{\dsd{inr}}{#1}}

\newcommand{\comp}[0]{\ensuremath{\mathbin{\circ}}}
\newcommand{\inv}[1]{\ensuremath{! \: #1}}

\newcommand{\Sonerec}[3]{\dsd{S^1\mathord{-}rec}_{#1}(#2,#3)}

\newcommand{\Srec}[4]{\dsd{S^{#1}\mathord{-}rec}_{#2}(#3,#4)}
\newcommand{\Selim}[4]{\dsd{S^{#1}\mathord{-}elim}_{#2}(#3,#4)}



\hyphenation{func-tor-ially}

\begin{document}

\title{A Cubical Approach to Synthetic Homotopy Theory}

% author names and affiliations
% use a multiple column layout for up to three different
% affiliations
\author{\IEEEauthorblockN{Daniel R. Licata
}
\IEEEauthorblockA{Wesleyan University\\
\url{dlicata@wesleyan.edu}}
\and
\IEEEauthorblockN{Guillaume Brunerie}
\IEEEauthorblockA{Université de Nice Sophia Antipolis \\
\url{guillaume.brunerie@gmail.com}}

\thanks{
This material is based on research sponsored by The United States Air
Force Research Laboratory under agreement number FA9550-15-1-0053. The
U.S. Government is authorized to reproduce and distribute reprints for
Governmental purposes notwithstanding any copyright notation thereon.
The views and conclusions contained herein are those of the authors and
should not be interpreted as necessarily representing the official
policies or endorsements, either expressed or implied, of the United
States Air Force Research Laboratory, the U.S. Government, or Carnegie
Mellon University.
}

}

\maketitle

\begin{abstract}
Homotopy theory can be developed synthetically in homotopy type theory, using types to describe spaces, the identity type to describe paths in a space, and iterated identity types to describe higher-dimensional paths. While some aspects of homotopy theory have been developed synthetically and formalized in proof assistants, some seemingly easy examples have proved difficult because the required manipulations of paths becomes complicated.  In this paper, we describe a cubical approach to developing homotopy theory within type theory.  The identity type is complemented with higher-dimensional cube types, such as a type of squares, dependent on four points and four lines, and a type of three-dimensional cubes, dependent on the boundary of a cube. Path-over-a-path types and higher generalizations are used to describe cubes in a fibration over a cube in the base.  These higher-dimensional cube and path-over types can be defined from the usual identity type, but isolating them as independent conceptual abstractions has allowed for the formalization of some previously difficult examples.
\end{abstract}

\section{Introduction}

\newcommand\Z{\ensuremath{\mathbb{Z}}}

\section{Introduction}

Homotopy type theory~\citep{...} is an extension of Martin-L\"of's
intensional type theory with new principles such as Voevodsky's
univalence axiom~\citep{...} and higher-dimensional inductive
types~\citep{...}.  These extensions are interesting both from a
computer science perspective, where they imbue the equality apparatus of
type theory with new computational meaning, and from a mathematical
perspective, where they allow higher-dimensional mathematics to be
expressed cleanly and elegantly in type theory.  One example of
higher-dimensional mathematics is the subject of homotopy theory, a
branch of algebraic topology.  In homotopy theory, one studies
topological spaces by way of their points, paths between points,
\emph{homotopies}, or paths between paths, and then homotopies between
homotopies, and so on.  This infinite tower of concepts---spaces,
points, paths, homotopies, and so on---is modeled in type theory by
types, elements of types, proofs of equality of elements, proofs of
equality of proofs of equality, and so on.  A space corresponds to a
type $A$. Points of a space correspond to elements $a,b : A$. Paths in a
space elements of the identity type (propositional equality), which we
notate $p : a =_A b$.  Homotopies between paths $p$ and $q$ correspond
to elements of the iterated identity type $p =_{a =_A b} q$.  The
general equality apparatus of type theory allows one to define all of
the operations and properties of paths.  These include identity paths
$\dsd{id} : a = a$ (reflexivity of equality), inverse paths $\inv p : b
= a$ when $p : a = b$ (symmetry of equality), and composition of paths
$q \comp p : a = c$ when $p : a = b$ and $q : b = c$ (transitivity of
equality), as well as homotopies relating these operations (for example,
$id \comp p = p$), and homotopies relating these homotopies, etc.  This
equips each type with the structure of a (weak)
\emph{$\infty$-groupoid}, as studied in higher category
theory~\citep{...}.  In category theoretic terminology, the elements of
a type correspond to objects (or 0-cells), the proofs of equality of
elements to morphisms (1-cells), the proofs of equality of proofs of
equality to 2-morphisms (2-cells), and so on.

One basic question in algebraic topology is calculating the
\emph{homotopy groups} of a space.  Given a space $X$ with a
distinguished point $x_0$, $\pi_1(X,x_0)$, the \emph{fundamental group},
is the group of loops at $x_0$, with composition as the group operation.
The fundamental group is the first in a sequence of \emph{homotopy
  groups}, which provide higher-dimensional information about a space:
the homotopy groups $\pi_n(X,x_0)$ ``count'' the $n$-dimensional loops in 
that space.  $\pi_2(X,x_0)$ is the group of homotopies between
$\dsd{id}_{{x_0}}$ and itself, $\pi_3(X,x_0)$ is the group of homotopies
between $\dsd{id}_{\dsd{id}_{{x_0}}}$ and itself, and so on.
\emph{Calculating a homotopy group} means proving a group isomorphism
between $\pi_n$ of a space and some explicit description of a group,
such as \Z\/ or $\Z_k$ (\Z\/ mod $k$).  

The homotopy groups of a space can be difficult to calculate.  This is
true even for spaces as simple as the $n$-dimensional spheres (the
circle, the sphere, \ldots)---some homotopy groups of spheres are
currently unknown.  A category-theoretic explanation for this fact is
the spheres (up to homotopy) can be presented as \emph{free
  $\infty$-groupoids} constructed by certain generators, and it can be
difficult to relate a presentation of a space as a free
$\infty$-groupoid to an explicit description of its homotopy groups.
For example, the circle is the free $\infty$-groupoids with one point
and one loop:

[fig]

\dsd{base} is a point (object) on the circle, and \dsd{loop} is a path
(morphism) from \dsd{base} to itself.  That the circle is the free
$\infty$-groupoid on these generators means that all the points, paths,
homotopies, \ldots on the circle are constructed by applying the
$\infty$-groupoid operations to these generators.  The generator
\dsd{loop} represents ``going around the circle once
counter-clockwise.''  Using the groupoid operations, one can construct
additional paths, such as $\inv {\dsd{loop}}$ (going around the circle once
clockwise) and $\dsd{loop} \comp \dsd{loop}$ (going around the circle
twice counter-clockwise).  Moreover, there are homotopies between paths,
such as $\dsd{loop} \comp \inv {\dsd{loop}} = \dsd{id}$ (going clockwise and
then counter-clockwise is the same, up to homotopy, as not going
anywhere at all).  In this case, one can calculate that, up to homotopy,
all loops on the circle are either \dsd{id} or $(\dsd{loop} \comp
\dsd{loop} \ldots \comp \dsd{loop})$ ($n$ times, for any $n$) or $(\inv{
\dsd{loop}} \comp \inv{  \dsd{loop}} \ldots \comp \inv{ \dsd{loop}})$ ($n$ times,
for any $n$), and thus that the loops on the circle are in
correspondence with the integers.  Thus, the fundamental group of the
circle is \Z.

However, in general, it can be quite difficult to relate a presentation
of a space as a free $\infty$-groupoid to an explicit description of its
homotopy groups, in part because of \emph{action across levels}.  For
example, the sphere can be presented as the free $\infty$-groupoid with
one point (0-cell) \dsd{base} and one homotopy (2-cell) $\dsd{loop}_2$
between $\dsd{id}_{{\dsd{base}}}$ (the path that stands still at
\dsd{base}) and itself---think of $\dsd{loop}_2$ as ``going around the
the surface of the sphere.''  An $\infty$-groupoid has group operations
at each level, so just as we have identity, inverse, and composition
operations on paths (1-cells), we have identity, inverse, and
composition operations on homotopies (2-cells).  Thus, we can form
homotopies such as \dsd{loop_2} \comp \dsd{loop_2} (going around the
surface of the sphere twice) and \dsd{loop_2} \comp \inv{\dsd{loop_2}}
(going around the surface of the sphere once in one direction, and then
in the opposite direction)---and, analgously to above, there is a
homotopy-between-homotopies relating the latter path to the constant
homotopy ($\dsd{loop_2} \comp \inv{\dsd{loop_2}} =
\dsd{id}_{{\dsd{id}_{\dsd{base}}}}$).  Thus, one would expect that the
homotopies (2-cells) on the sphere have the same structure as the paths
(1-cells) on the circle, and this is indeed the case: $\pi_2(S^2)$ is
also \Z.  However, $\infty$-groupoids have more structure than just the
group operations at each level---for example, lower-dimensional
generators can construct higher-dimensional paths.  An example of this
is that $\pi_3(S^2)$, the group of homotopies between homotopies
(3-cells) on the sphere, is also \Z, \emph{despite the fact that there
  are no generators for 3-cells in the presentation of the sphere!}  The
paths arise from the applying the algebraic operations of a
$\infty$-groupoid to the 2-cell generator $\dsd{loop_2}$---and this
action across levels is one reason that homotopy groups are so difficult
to calculate.  

One enticing idea is to use homotopy type theory to calculate homotopy
groups: by doing so, we can give computer-checked proofs of these
calculations, and we can potentially exploit constructivity and the
type-theoretic perspective $\infty$-groupoids to attack these difficult
problems in algebraic topology.  Fortunately, it is simple to pose the
problem of calculating a homotopy group in homotopy type theory.  First,
we describe a space using \emph{higher inductive types}, which
generalize ordinary inductive types by allowing constructors not only
for elements of the type, but for paths (proofs of equality).  For
example, the circle is represented by a higher inductive type with two
constructors

\[
\begin{array}{l}
\dsd{base} : S^1 \\
\dsd{loop} : \dsd{base} =_{S^1} \dsd{base} \\
\end{array}
\]

This says that \dsd{base} is a point on the circle, while \dsd{loop} is
a path from \dsd{base} to \dsd{base}.  




Assume familiarity with LICS paper and HoTT book.  




\input{PathOver.lao}
\input{Square.lao}
\input{Cube.lao}
\input{TS1S1.lao}
%% \input{TS1S1-full.lao}
\input{3x3.lao}
\section{Related and Future Work}

The various definitions of path-over, and their equivalence, were
discussed during the IAS special year (see \citep[Remark
  6.3.2]{uf13hott-book}).  The idea to use inductive families to define
shapes other than lines was explored in a simplicial setting by Coquand
(e.g. a triangle in a type).  At the time, it was considered an open
question whether it was helpful to consider shapes other than the
homogeneous globes that arise by iterating the identity type.  In this
paper, we have argued that there are benefits to working with cube and
cube-over types.  Higher cubes and cube-overs arise naturally when
higher inductive types have two-dimensional (or higher) constructors
(like the torus), when higher inductive types are nested (a pushout of
pushouts, like in the three-by-three lemma), and when eliminations on
higher inductive types are nested (like when mapping out of two
circles).  Developing lemmas in terms of these abstractions has enabled
the new formalizations described in this paper.

At present, we have developed cube types in ``offshoots'' of Agda
homotopy type theory libraries; an interesting direction for future work
would be to revise the libraries to use cubes throughout, and to revisit
existing results to see if they can be simplified.  Another direction
for future work is to define higher cube and cube-over types in a
dimension-polymorphic way, rather than implementing each dimension in
isolation, as we have done here.

Our work is carried out in the setting of dependent type theory with
axioms for univalence and higher inductives, but there have also been
extensions of dependent type theory based on cubical ideas.  These
include a type theory with a computational interpretation of
parametricity~\citep{bernardy12parametricity}, and new cubical type
theories~\citep{coquand14variations,altenkirchkaposi14cubical,lb14cubes-oxford}
inspired by the cubical sets model of homotopy type
theory~\citep{coquand+13cubical}.  These cubical type theories add new
definitional equalities to the axiomatic formulations of univalence and
higher inductives.  The examples presented here could be translated to
cubical type theories, to investigate the impact of these new
definitional equalities.  Additionally, it is an open question whether
results in these stricter cubical type theories can be translated back
to the axiomatic type theory, by inserting more propositional reasoning;
the low-dimensional libraries and examples presented here provide a bit
of empirical evidence in the positive direction.


\medskip

\textbf{Acknowledgments} We thank Joseph Lee, who attempted
to prove the torus-circles equivalence with the first author in summer
2012.  We thank the participants of the IAS year on univalent
foundations and the Paris trimester on semantics of proofs and certified
mathematics for many helpful discussions.  We thank the anonymous
reviewers for helpful feedback.  

\setlength{\bibsep}{-1pt} %% dirty trick: make this negative
{ \small
%% \linespread{0.70}
\bibliographystyle{abbrvnat}
\bibliography{drl-common/cs}
}

\end{document}

