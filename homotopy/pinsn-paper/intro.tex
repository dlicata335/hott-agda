\section{Introduction}

Homotopy type theory~\citep{...} is an extension of Martin-L\:of's
intensional type theory with new principles such as Voevodsky's
univalence axiom~\citep{...} and higher-dimensional inductive
types~\citep{...}.  These extensions are interesting both from a
computer science perspective, where they imbue the equality apparatus of
type theory with new computational meaning, and from a mathematical
perspective, where they allow higher-dimensional mathematics to be
expressed cleanly and elegantly in type theory.  One example of
higher-dimensional mathematics is the subject of homotopy theory, a
branch of algebraic topology.  In homotopy theory, one studies
topological spaces by way of their points, paths between points,
\emph{homotopies}, or paths between paths, and then homotopies between
homotopies, homotopies between homotopies, and so on.  This infinite
tower of concepts---spaces, points, paths, homotopies, \ldots---is
modeled in type theory by types, elements of types, proofs of equality
of elements, proofs of equality of proofs of equality, and so on.  A
space corresponds to a type $A$. Points of a space correspond to
elements $a,b : A$. Paths in a space elements of the identity type, or
propositional equality, which we notate $p : a =_A b$.  Homotopies
between paths $p$ and $q$ correspond to elements of the iterated
identity type $p =_{a =_A b} q$.  The general equality apparatus of type
theory allows one to define all of the operations and properties of
paths.  These include identity paths $\dsd{id} : a = a$ (reflexivity of
equality), inverse paths $! p : b = a$ when $p : a = b$ (symmetry of
equality), and composition of paths $q \circ p : a = c$ when $p : a = b$ and
$q : b = c$ (transitivity of equality), as well as homotopies relating
these operations (for example, $id \circ p = p$), and homotopies
relating these homotopies, etc.  This equips each type with the
structure of an \emph{$\infty$-groupoid}~\citep{...}.  

Basic topological spaces are defined using \emph{higher inductive
  types}, which generalize ordinary inductive types by allowing
constructors not only for elements of the type, but for paths (proofs of
equality).  For example, a circle

[fig]

is represented by a higher inductive type with two constructors

\[
\begin{array}{l}
\dsd{base} : S^1 \\
\dsd{loop} : \dsd{base} =_{S^1} \dsd{base} \\
\end{array}
\]

This says that \dsd{base} is a point on the circle, while \dsd{loop} is
a path from \dsd{base} to \dsd{base}.  The groupoid equips $S^1$ with
operations such as inverses (e.g. $! \dsd{loop}$) and composition
(e.g. $\dsd{loop} \circ \dsd{!  loop}$), and homotopies (e.g. inverses
cancel--- $\dsd{loop} \circ ! \dsd{loop} = \dsd{id}$).  

Assume familiarity with LICS paper and HoTT book.  


higher-dimensional mathematics, we mean subjects such as homotopy theory

, we mean subjects such as homotopy theory and category
theory, which study infinite
