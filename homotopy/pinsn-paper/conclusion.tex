
\section{Conclusion}

In this paper, we have described a computer-checked calculation of
$\pi_n(S^n)$ in homotopy type theory.  One important direction for
future work is to develop a computational interpretation of homotopy
type theory; our proof would be a good test case for such an
interpretation. Given a number $k$, how does the proof compute the path
$\dsd{loop}_n^k$?  Or, more interestingly, given a path on $S^n$, how
does the proof compute a number?  Another direction would be to
investigate the relationship between this proof and proofs of
$\pi_n(S^n)$ in classical homotopy theory.  The proof we have described
here has since been generalized to a proof of the Freudenthal Suspension
Theorem~\citep{uf13hott-book}, which is one way that $\pi_n(S^n)$ is proved in classical
homotopy theory.  However, it would be interesting to see whether the
more specific proof presented here has
been (or can be) phrased in classical terms.  
